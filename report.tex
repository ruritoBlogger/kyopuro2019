\documentclass[11pt,a4paper]{jsarticle}
%
\usepackage{amsmath,amssymb}
\usepackage{bm}
\usepackage{graphicx}
\usepackage{ascmac}
%
\setlength{\textwidth}{\fullwidth}
\setlength{\textheight}{40\baselineskip}
\addtolength{\textheight}{\topskip}
\setlength{\voffset}{-0.2in}
\setlength{\topmargin}{0pt}
\setlength{\headheight}{0pt}
\setlength{\headsep}{0pt}
%
\newcommand{\writtenBy}[1]{\begin{flushright}文責: #1\end{flushright}~}
\newcommand{\divergence}{\mathrm{div}\,}  %ダイバージェンス
\newcommand{\grad}{\mathrm{grad}\,}  %グラディエント
\newcommand{\rot}{\mathrm{rot}\,}  %ローテーション
%
\title{競技プログラミング班 活動報告書}
\author{服部瑠斗 小村漱一朗 中野海人 稲垣和真 浜田直弥 西見元希 石川琉聖}
\date{\today}
\begin{document}
\maketitle
%
%
\section{活動の概要}
\writtenBy{稲垣 和真}
本プロジェクトは、競技プログラミングを通してプログラミングにおけるアルゴリ
ズムの知見を深めるために発足した。競技プログラミングに用いたツールはAtCoderであ
る。AtCoderのBeginner Contestや班内の活動内で開催したバーチャルコンテストを通し
て実際に問題を解き、その解法を活動で共有し、お互いの知見を高めたり、いろいろな
アルゴリズムの手法について会員が解説し、理解を深めるというものである

\section{競技プログラミングについて}
\writtenBy{稲垣和真}
競技プログラミングとは、解くべき問題が与えられ、その問題を解くプログラムを、早
く正確に解き、正解数や解くまでにかかった時間などを競う競技である。
\section{学習内容}
\subsection{アルゴリズム}
\subsection{競技プログラミングにおけるテクニック}
\section{活動で得られたもの}
\section{問題点}
\section{展望}


%
%
\end{document}
