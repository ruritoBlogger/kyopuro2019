\documentclass[11pt,a4paper]{jsarticle}
%
\usepackage{amsmath,amssymb}
\usepackage{bm}
\usepackage{graphicx}
\usepackage{ascmac}
%
\setlength{\textwidth}{\fullwidth}
\setlength{\textheight}{40\baselineskip}
\addtolength{\textheight}{\topskip}
\setlength{\voffset}{-0.2in}
\setlength{\topmargin}{0pt}
\setlength{\headheight}{0pt}
\setlength{\headsep}{0pt}
%
\newcommand{\divergence}{\mathrm{div}\,}  %ダイバージェンス
\newcommand{\grad}{\mathrm{grad}\,}  %グラディエント
\newcommand{\rot}{\mathrm{rot}\,}  %ローテーション
%
\title{競技プログラミング班 活動報告書}
\author{服部瑠斗 小村漱一朗 中野海人 稲垣和真 浜田直弥 西見元希 石川琉聖}
\date{\today}
\begin{document}
\maketitle
%
%
\section{活動の概要}
\section{競技プログラミングについて}
\section{学習内容}
\subsection{アルゴリズム}
\subsection{競技プログラミングにおけるテクニック}
\section{活動で得られたもの}
\section{問題点}
\writtenBy{西見 元希}
  このプロジェクトにおける問題点として、まず集まりの悪さが挙げられる。そもそも活動時間を班員全員が問題なく出席できる時間に設定することができなかったために班員が十分に出席できず、結果として活動は数回のバーチャルコンテストと解説にとどまることになってしまった。これは今後のプロジェクト活動において改善すべき問題点であると考えられる。具体的な解決案としては現在いくつかのプロジェクトで見られる週数回の活動日を設定し、同じ内容を学習するというものが挙げられる。
\section{展望}

%
%
\end{document}
